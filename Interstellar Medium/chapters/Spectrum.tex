\chapter{Spectral lines}
The spectroscopic notation is a way to describe the energy levels of an atom or a molecule,
\begin{definition}[Spectroscopic notation]
\begin{equation}
  ^{\eqnmarkbox[blue]{S+1}{2S+1}}\eqnmarkbox[red]{L}{L}_{\eqnmarkbox[PineGreen]{J}{J}}^{\eqnmarkbox[orange]{p}{(p)}}.
\end{equation}
\annotate[yshift=-0.5em]{below, left}{S+1}{Spin multiplicity}
\annotate[yshift=-1.5em]{below, left}{L}{Orbital angular momentum}
\annotate[yshift=-0.5em]{below, right}{J}{Total angular momentum}
\annotate[yshift=1em]{above, right}{p}{Term parity}
\end{definition}
The parity characterizes whether the wave function changes sign under reflection of all electron positions through the origin. It is blank for even parity and $o$ for odd parity. The selection rules for the transitions are, 
\begin{itemize}
  \item Parity must change 
  \item $\Delta L = 0, \pm 1$ 
  \item $\Delta J = 0, \pm 1$, but $\Delta J = 0\rightarrow 0$ is forbidden
  \item Only one single electron wavefunction $n\ell$ can change with $\Delta\ell = \pm 1$ 
  \item $\Delta S = 0$ for electric dipole transitions
\end{itemize} 
This approximation is valid when radiation involved has $\lambda\gg a_0$.
Each electron is affected by the electric field produced by all other charges within the atom, 
\begin{equation}
  D_{ki} = \int \psi^*_k(\mathbf{r})\eqnmarkbox[red]{dipole}{\mathbf{D}}\psi_i(\mathbf{r})\,d\mathbf{r}.
\end{equation}
\annotate[yshift=-0.5em]{below, right}{dipole}{Dipole approximation}\\

\vspace{-5mm}
\textsf{Forbidden lines} are when one of the rules is not satisfied, while the \textsf{semi-forbidden lines} are where the spin selection rule is violated, and are much weaker than intersystem (permitted) transitions.
