\chapter{Classical Field Theory}
\section{Dynamics of fields}
\subsection{Fields and their properties}
A scalar field, named $a$ has a formula, 
\begin{equation}
  \phi_a(\mathbf{x}, t),
\end{equation}
like the electric and magnetic fields, $\mathbf{E}(\mathbf{x}, t)$, $\mathbf{B}(\mathbf{x}, t)$. As the space and time are now combined, the manifold has to be described using Minkowski spacetime.
\begin{definition}[Metric]
  \begin{equation}
    ds^2 = -(dx^0)^2 + (dx_1)^2 + (dx_2)^2 + (dx_3)^2,
  \end{equation}
  \vspace{-0.5cm}
\end{definition}

or can be rewritten as a four-vector,
\begin{equation}
  x^\mu = (x^0, x^1, x^2, x^3) = (x^0, \mathbf{x}),
\end{equation}
where $x^0 = ct$ and $x^i = x^i$ for $i = 1, 2, 3$. The metric tensor is then,
\begin{definition}[Metric tensor for Minkowski space]
\begin{equation}
  \eta_{\mu\nu} = \begin{bmatrix}
    -1 & 0 & 0 & 0 \\
    0 & 1 & 0 & 0 \\
    0 & 0 & 1 & 0 \\
    0 & 0 & 0 & 1 
  \end{bmatrix}, 
\end{equation}
\vspace{-0.5cm}
\end{definition}
giving the metric as, 
\begin{equation}
  ds^2 = \eta_{\mu\nu}dx^\mu dx^\nu = \eta^{\mu\nu}dx_\mu dx_\nu. 
\end{equation}\\

\begin{Note}[Properties of the metric tensor]
  \begin{equation}
    \eta^{-1}\eta = \mathds{1},\quad \eta^{\mu\nu}\eta_{\nu\rho} = \delta^\mu_\rho.
  \end{equation}
\end{Note}\vspace{0.5cm}

The covariant and contravariant vectors are related by the metric tensor,
\begin{equation}
  x_\mu = \eta_{\mu\nu}x^\nu, \quad x^\mu = \eta^{\mu\nu}x_\nu, 
\end{equation}
and another tensor can be defined as, 
\begin{equation}
  B^\mu = \eta^{\mu\nu}A_\nu, \quad A_\mu = \eta_{\mu\nu}B^\nu. 
\end{equation}

\begin{definition}[Contraction of tensors]
  \begin{equation}
    \eta_{\mu\alpha}T^{\alpha\nu\rho} = T\indices{_\mu^\nu^\rho}
  \end{equation}
  \vspace{-0.5cm}
\end{definition}
\vspace{0.5cm}
Returning back to the definitions of the electric and magnetic fields, a field can be written as 
\begin{equation}
  A^{\mu} (\mathbf{x}, t) = (\phi, \mathbf{A}),
\end{equation}
where $\phi$ is the scalar potential and $\mathbf{A}$ is the vector potential. The electric and magnetic fields are connected to it by
\begin{equation}
  \mathbf{E} = -\nabla \phi - \frac{\partial \mathbf{A}}{\partial t}, \quad \mathbf{B} = \nabla \times \mathbf{A}. 
\end{equation}

\subsection{Action and Lagrangian}
The action for a system is defined as,
\begin{definition}[Action]
  \begin{equation}
    S = \int d^4x\, \mathcal{L} (\phi_a, \partial_\mu \phi_a),
  \end{equation}
  \vspace{-0.5cm}
\end{definition}

\begin{Note}[Partial derivatives]
  \begin{equation}
    \partial_\mu = \frac{\partial}{\partial x^\mu} = \left(\frac{\partial}{\partial t}, \nabla\right). 
  \end{equation}
\end{Note}

A field is defined with no variation at the endpoints, $\delta \phi_a = 0$, and the action is stationary. Varying the action gives, 
\begin{equation}
  \delta S = \int d^4x\, \left(\frac{\partial \mathcal{L}}{\partial \phi_a}\delta \phi_a + \frac{\partial \mathcal{L}}{\partial (\partial_\mu \phi_a)}\delta (\partial_\mu \phi_a)\right).
\end{equation}

Using the property, 
\begin{lemma}
  \begin{equation}
    \delta (AB) = A\,\delta B + B\,\delta A,
  \end{equation}
  \vspace{-0.5cm}
\end{lemma}
the variation of the action can be written as,
\begin{equation}
  \delta S = \int d^4x\, \left(\frac{\partial \mathcal{L}}{\partial \phi_a} - \partial_\mu \left(\frac{\partial \mathcal{L}}{\partial (\partial_\mu \phi_a)}\right)\right)\delta \phi_a + \partial_\mu \left(\frac{\partial \mathcal{L}}{\partial (\partial_\mu \phi_a)}\delta \phi_a\right).
\end{equation}

As the last term is a total derivative, it becomes 0 as it vanishes at the endpoints. The Euler-Lagrange equation is then,
\newpage
\begin{definition}[Euler-Lagrange equation]
  \begin{equation}
    \frac{\partial \mathcal{L}}{\partial \phi_a} - \partial_\mu \left(\frac{\partial \mathcal{L}}{\partial (\partial_\mu \phi_a)}\right) = 0.
  \end{equation}
  \vspace{-0.5cm}
\end{definition}
\subsubsection{Maxwell's equations}
For example, taking the Lagrangian of the electromagnetic field, 
\begin{equation}
  \mathcal{L} = -\frac{1}{2}(\partial_\mu A_\nu)(\partial^\mu A^\nu) + \frac{1}{2}(\partial_\mu A^\mu)^2,
\end{equation}
the Euler-Lagrange equation is, 
\begin{equation}
  \partial_\mu \left(\frac{\partial \mathcal{L}}{\partial (\partial_\mu A_\nu)}\right) - \frac{\partial \mathcal{L}}{\partial A_\nu} = 0,
\end{equation}

\begin{Question}
  Derive the Maxwell's equations for the electromagnetic field, given the Lagrangian
  \begin{equation}
    \mathcal{L} = -\frac{1}{2}(\partial_\mu A_\nu)(\partial_\alpha A_\beta)\eta^{\mu\alpha}\eta^{\nu\beta} + \frac{1}{2}(\partial_\mu A^\mu)^2.
  \end{equation}
\end{Question}
\begin{lemma}[Kronecker delta]
  \begin{equation}
    \frac{\partial A_\mu }{\partial A_\nu} = \delta^\nu_\mu \quad \partial_\mu A^\nu = \delta^\nu_\mu
  \end{equation}
  \vspace{-0.5cm}
\end{lemma}
  
Knowing that there is no $A_\nu$ dependence as all $A_\nu$ terms are in the derivative, the Euler-Lagrange equation simplifies to,
\begin{equation}
  \partial_\sigma \left(\frac{\partial \mathcal{L}}{\partial (\partial_\sigma A_\tau)}\right) = 0,
\end{equation}
and the Maxwell's equations are derived as,
\begin{derivation}
\begin{equation}
  \begin{aligned}
    -\frac{1}{2} \partial_\sigma \eta^{\mu\alpha}\eta^{\nu\beta} \left(\partial_\mu A_\nu \frac{\partial(\partial_\alpha A_\beta)}{\partial (\partial_\sigma A_\tau)} + \frac{\partial(\partial_\mu A_\nu)}{\partial (\partial_\sigma A_\tau)}\partial_\alpha A_\beta\right) + \partial_\sigma\left(\partial_\mu A^\mu \eta^{\mu\nu}\frac{\partial(\partial_\mu A_\nu)}{\partial (\partial_\sigma A_\tau)}\right) &= 0\\
    -\frac{1}{2} \partial_\sigma \eta^{\mu\alpha}\eta^{\nu\beta} \left(\partial_\mu A_\nu \delta^\sigma_\alpha \delta^\tau_\beta + \delta^\sigma_\mu \delta^\tau_\nu \partial_\alpha A_\beta\right) +  \partial_\sigma \eta^{\mu\nu} \left(\partial_\mu A^\mu \delta^\sigma_\mu \delta^\tau_\nu\right) &= 0\\
    -\frac{1}{2} \partial_\sigma \left(\eta^{\mu\sigma}\eta^{\nu\tau}\partial_\mu A_\nu + \eta^{\sigma\alpha}\eta^{\tau\beta}\partial_\alpha A_\beta\right) +  \partial_\sigma \left(\eta^{\sigma\tau}\partial_\mu A^\mu\right) &= 0\\
   -\partial_\sigma \partial^\sigma A^\tau + \partial^\tau \partial_\mu A^\mu &= 0\\
   \text{reindexing }\sigma\rightarrow\mu, \tau\rightarrow\nu \quad -\partial_\mu (\partial^\mu A^\nu - \partial ^\nu A^\mu) &= 0\\
   -\partial_\mu F^{\mu\nu} &= 0,
  \end{aligned}
\end{equation}
\end{derivation}
where $F_{\mu\nu} = \partial_\mu A_\nu - \partial_\nu A_\mu$ is the \textsf{field strength tensor}.
\subsubsection{Klein-Gordon equation}
Another example is the Klein-Gordon equation,
\begin{equation}
\begin{aligned}
  \mathcal{L} &= \frac{1}{2}(\partial_\mu \phi)(\partial^\mu \phi) - \frac{1}{2}m^2\phi^2\\
              &= \frac{1}{2}\eta^{\mu\nu}(\partial_\mu \phi)(\partial_\nu \phi) - \frac{1}{2}m^2\phi^2\\
              &= \eqnmarkbox[red]{T}{\frac{1}{2} \dot \phi^2} - \eqnmarkbox[blue]{V}{\frac{1}{2}(\nabla \phi)^2 - \frac{1}{2}m^2\phi^2},
\end{aligned}
\end{equation}
\annotate[yshift=-0.5em]{below, left}{T}{$T=\int d^3x\frac{1}{2} \dot \phi^2$}
\annotate[yshift=-0.5em]{below, right}{V}{$V=\int d^3x\left(\frac{1}{2}(\nabla \phi)^2 + \frac{1}{2}m^2\phi^2\right)$}\\

\vspace{-3mm}
which is a Lagrangian for a scalar field. To find the equations of motion, the Euler-Lagrange equation is used,
\begin{equation}
  \frac{\partial\mathcal{L}}{\partial\phi} = -m^2\phi \quad \frac{\partial\mathcal{L}}{\partial(\partial_\mu \phi)} = \partial^\mu \phi,
\end{equation}
\begin{derivation}
  \begin{equation}
    \begin{aligned}
      \frac{\partial\mathcal{L}}{\partial(\partial_\sigma\phi)} &= \frac{1}{2}\eta^{\mu\nu}\left(\partial_\mu\phi \frac{\partial(\partial_\nu\phi)}{\partial(\partial_\sigma\phi)} + \frac{\partial(\partial_\mu\phi)}{\partial(\partial_\sigma\phi)}\partial_\nu\phi\right)\\
                                                             &= \frac{1}{2}\eta^{\mu\nu}(\partial_\mu\phi \delta^\sigma_\nu + \delta^\sigma_\mu\partial_\nu\phi)\\
                                                             &= \frac{1}{2}(\eta^{\mu\sigma}\partial_\mu\phi + \eta^{\sigma\nu}\partial_\nu\phi)\\
                                                             &=\partial^\sigma\phi,
    \end{aligned}
  \end{equation}
\end{derivation}

and the Klein-Gordon equation is derived as,
\begin{equation}
  \partial_\mu \partial^\mu \phi + m^2\phi = 0. 
\end{equation}

\subsubsection{First Order Lagrangians}
Consider a Lagrangian that is linear in time derivatives, 
\begin{equation}
  \mathcal{L} = \frac{i}{2}(\psi^*\dot\psi - \dot\psi^*\psi) - \nabla\psi^*\cdot\nabla\psi - m\psi^*\psi,
\end{equation}
treating $\psi$ and $\psi^*$ as independent variables. The Euler-Lagrange equations are then,
\begin{equation}
  \frac{\partial\mathcal{L}}{\partial\psi^*} = \frac{i}{2}\dot\psi - m\psi \quad \frac{\partial\mathcal{L}}{\partial\psi^*} = -\frac{i}{2}\psi \quad \frac{\partial\mathcal{L}}{\partial\nabla\psi^*} = -\nabla\psi,
\end{equation}
giving the equation of motion, 
\begin{equation}
  i\dot\psi = -\nabla^2\psi +m\psi.
\end{equation}
Compared to the Klein-Gordon equation, only $\psi$ and $\psi^*$ are needed to determine the future evolution compared to needing $\phi$ and $\dot\phi$.

\subsubsection{Note on locality}
All examples above, as well as all thories of nature are local. This means that there are no terms in the Lagrangian that look like, 
\begin{equation}
  \mathcal{L} = \int d^3x\,d^3y\phi(\mathbf{x})\phi(\mathbf{y}). 
\end{equation}
One of the main reasons for introducing field theories in classical physics is to account for locality.

\section{Lorenz invariance}

Since relativity has to be accounted for, the field theory has to be invariant under Lorentz transformations, i.e., 
\begin{equation}
  x^\mu \rightarrow (x')^{\mu} = \Lambda\indices{^\mu_\nu}x^\nu,
\end{equation}
where, 
\begin{lemma}[Lorentz tranformation]
  \begin{equation}
    \Lambda\indices{^\mu_\sigma}\eta^{\sigma\tau}\Lambda\indices{^\nu_\tau} = \eta^{\mu\nu}.
  \end{equation}
  \vspace{-0.5cm}
\end{lemma}

The field has to transform as well, a simple example is a scalar field which transforms as,
\begin{Note}[Scalar field transformation]
\begin{equation}
  \phi(x) \rightarrow \phi'(x) = \phi(\Lambda^{-1}x),
\end{equation}
\end{Note}
and the derivative of a scalar field transforms as, 
\begin{Note}[Derivative of scalar field transformation]
\begin{equation}
  \partial_\mu \phi(x) \rightarrow (\Lambda^{-1})\indices{^\nu_\mu}\partial_\nu \phi(\Lambda^{-1}x).
\end{equation}
\end{Note}
The inverse transformation is used as coordinates are not relabeled after the transformation. From the \textsf{Lorentz invariant theory}, if $\phi(x)$ is a solution to the equations of motion, then $\phi(\Lambda^{-1}x)$ is also a solution. 
\subsubsection{Maxwell's equations}
Using the Lorentz transformation $A^\mu(x) \rightarrow \Lambda\indices{^\mu_\nu}A^\nu(\Lambda^{-1}x)$, 
\begin{derivation}
  \begin{equation}
    \begin{aligned}
      (\partial_\mu A_\nu(x))(\partial_\beta A^\nu(x))\eta^{\mu\beta} &\rightarrow (\partial_\mu\eta_{\nu\alpha}\Lambda\indices{^\alpha_\gamma}\eta^{\gamma\epsilon}A_\epsilon(\Lambda^{-1}x))( \partial_\beta\Lambda\indices{^\nu_\delta}\eta^{\delta\zeta}A_\zeta(\Lambda^{-1}x))\eta^{\mu\beta}\\
                                                                      &= \eta_{\nu\alpha}\Lambda\indices{^\alpha_\gamma}\eta^{\gamma\epsilon}\Lambda\indices{^\nu_\delta}\eta^{\delta\zeta}(\partial_\mu A_\epsilon(\Lambda^{-1}x))(\partial_\beta A_\zeta(\Lambda^{-1}x))\eta^{\mu\beta}\\
                                                                      &= \Lambda\indices{_\nu^\epsilon}\Lambda\indices{^\nu^\zeta}(\partial_\mu A_\epsilon(\Lambda^{-1}x))(\partial_\beta A_\zeta(\Lambda^{-1}x))\eta^{\mu\beta}\\
                                                                      &= (\partial_\mu A_\epsilon(\Lambda^{-1}x))(\partial_\beta A_\zeta(\Lambda^{-1}x))\eta^{\mu\beta}\eta^{\epsilon\zeta}\\
                                                                      &= (\partial_\mu A_\epsilon(\Lambda^{-1}x))(\partial_\beta A^\epsilon(\Lambda^{-1}x))\eta^{\mu\beta},
    \end{aligned}
  \end{equation}

\end{derivation}
where the derivation used, 
\begin{derivation}
  \begin{equation}
    \begin{aligned}
      \Lambda\indices{^\mu_\alpha}\Lambda\indices{^\nu_\beta}\eta_{\mu\nu} &= \eta_{\alpha\beta}\\
      \eta^{\alpha\epsilon}\eta^{\beta\zeta}\Lambda\indices{^\mu_\alpha}\Lambda\indices{^\nu_\beta}\eta_{\mu\nu} &= \eta^{\alpha\epsilon}\eta^{\beta\zeta}\eta_{\alpha\beta}\\ 
      \Rightarrow \quad \Lambda\indices{_\nu^\epsilon}\Lambda\indices{^\nu^\zeta} &= \eta^{\epsilon\zeta}.
    \end{aligned}
  \end{equation}
\end{derivation}
For the second part of the Lagrangian,
\begin{derivation}
  \begin{equation}
    \begin{aligned}
      \partial_\mu A^\mu(x) &\rightarrow (\Lambda^{-1})\indices{^\alpha_\mu}\partial_\alpha \Lambda\indices{^\mu_\beta}A^\beta(\Lambda^{-1}x)\\
                           &= \delta^\alpha_\beta\partial_\alpha A^\beta(\Lambda^{-1}x)\\
                           &= \partial_\alpha A^\alpha(\Lambda^{-1}x),
    \end{aligned}
  \end{equation}
\end{derivation}
showing that Maxwell's equations are \hyellow{Lorentz invariant}.

\subsubsection{Klein-Gordon equation}
The Klein-Gordon equation \hyellow{is Lorentz invariant}, as the derivatives transform as,
\begin{derivation}
  \begin{equation}
    \begin{aligned}
      \partial_\mu \phi(x)\partial_\nu\phi(x)\eta^{\mu\nu} &\rightarrow (\Lambda^{-1})\indices{^\rho_\mu}\partial_\rho \phi(\Lambda^{-1}x)(\Lambda^{-1})\indices{^\sigma_\nu}\partial_\sigma\phi(\Lambda^{-1}x)\eta^{\mu\nu}\\
                                                           &= \partial_\rho \phi(\Lambda^{-1}x)\partial_\sigma\phi(\Lambda^{-1}x)\Lambda\indices{^\rho_\mu}\eta^{\mu\nu}\Lambda\indices{^\sigma_\nu}\\
                                                           &= \partial_\rho \phi(\Lambda^{-1}x)\partial_\sigma\phi(\Lambda^{-1}x)\eta^{\rho\sigma}
    \end{aligned}
  \end{equation}
\end{derivation}


The action is then Lorentz invariant, however there is no Jacobian factor in the integral after change of variables due to $\text{det}\,\Lambda=1$.
\subsubsection{First Order Lagrangians}
Since the Lagrangian is linear in time derivative, but quadratic in spatial derivatives, the theory is \hyellow{not Lorentz invariant}.

\section{Symmetries}

