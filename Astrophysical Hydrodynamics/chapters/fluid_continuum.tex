\chapter{Fluids as continuums}
\section{Fluid element}
A \textsf{fluid element} is a portion of the fluid of size $l$ that is large compared to the mean free path $\ell$, but small compared to the characteristic length scale of the system $L$,
\begin{equation}
  \ell \ll l \ll L. 
\end{equation}
The length scale is defined as
\begin{equation}
  L\sim\frac{q}{\nabla q}
\end{equation}
where $q$ is a macroscopic physical quantity. In a CNM, the ``worst case scenario'', $A = A(\text{H},\text{H})\approx 10^{-15}\,\text{cm}^{2}$, and $n\approx30\,\text{cm}^{-3}$, so $\ell\approx10^{13}\,\text{cm}$. This means that any structure in the ISM would have an interaction that is smaller than this. 

\section{Advection}
The \textsf{advection} of a fluid element is the transport of the element by the flow. Consider a fluid element moving with velocity $\mathbf{u}$ and has a generic quantity $q$ which is a function of position and time, $q(\mathbf{r},t)$. The change in $q$ is given by the \textsf{Lagrangian derivative},
\begin{equation}
  \frac{\text{D}q}{\text{D}t} = \frac{\partial q}{\partial t} + \mathbf{u}\cdot\nabla q. 
\end{equation}
It is explicitly defined as, 
\begin{equation}
  \begin{aligned}
    \frac{\text{D}q}{\text{D}t} &=\lim_{\delta t\to0}\frac{q(\mathbf{r}+\delta \mathbf{r},t+\delta t) - q(\mathbf{r},t)}{\delta t}\\ 
                  &= \lim_{\delta t\to0}\frac{q(\mathbf{r}+\delta \mathbf{r},t+\delta t) - q(\mathbf{r},t) + q(\mathbf{r}, t+\delta t) - q(\mathbf{r}, t+\delta t)}{\delta t}\\
                  &=  \left(\frac{\partial q}{\partial t}\right)_{(\mathbf{r}, t)} + \lim_{\delta t\to0}\frac{q(\mathbf{r}+\delta \mathbf{r},t+\delta t)  - q(\mathbf{r}, t+\delta t)}{\delta t}\\
                  &= \left(\frac{\partial q}{\partial t}\right)_{(\mathbf{r}, t)} + \lim_{\delta t\to0}\frac{ \delta\mathbf{r}\cdot(\nabla q)_{(\mathbf{r}, t)}}{\delta t}\\
                  &= \frac{\partial q}{\partial t} + \mathbf{u}\cdot\nabla q.
  \end{aligned}
\end{equation}
Taking $q$ to be a generic quantity, 
\begin{definition}[Lagrangian derivative]
\begin{equation}
  \frac{\text{D}}{\text{D}t} = \eqnmarkbox[blue]{euler}{\frac{\partial}{\partial t}} + \eqnmarkbox[red]{advection}{\mathbf{u}\cdot\nabla}. 
\end{equation}
\annotate[yshift=-0.5em]{below}{euler}{Eulerian derivative}
\annotate[yshift=+0.5em]{above}{advection}{Advection operator}
\vspace{0cm}
\end{definition}

\section{Continuity equation}
Consider a portion of fluid of density $\rho$ constained in a volume $V$. The mass inside the volume is, 
\begin{equation}
  M = \int_V \rho\,dV. 
\end{equation} 
The change in mass is given by the rate of change of mass inside the volume,
\begin{equation}
  \frac{\partial M}{\partial t} = \frac{\partial}{\partial t}\int_V \rho\,dV = -\oint_S \rho\mathbf{u}\cdot d\mathbf{S},
\end{equation} 
where $S$ is the surface of the volume $V$. Using the divergence theorem, 
\begin{theorem}[Divergence theorem]
\begin{equation}
  \oint_S \mathbf{F}\cdot d\mathbf{S} = \int_V \nabla\cdot\mathbf{F}\,dV, 
\end{equation}
\vspace{-0.5cm}
\end{theorem}
the equation becomes, 
\begin{equation}
  \begin{aligned}
    \frac{\partial}{\partial t}\int_V \rho\,dV &= -\oint_S \rho\mathbf{u}\cdot d\mathbf{S}\\
                                  &= -\int_V \nabla\cdot(\rho\mathbf{u})\,dV\\
                                  &\Rightarrow \frac{\partial \rho}{\partial t} + \nabla\cdot(\rho\mathbf{u}) = 0,
  \end{aligned}
\end{equation}
which is the \textsf{continuity equation} or the \textsf{mass conservation equation}. It can be rewritten as, 
\begin{equation}
  \frac{\text{D}\rho}{\text{D}t} + \rho\nabla\cdot\mathbf{u} = 0. 
\end{equation}
For an \textsf{incompressible fluid}, $\nabla\cdot\mathbf{u} = 0 \Leftrightarrow \text{D}\rho/\text{D}t = 0$.

\section{Euler equation}
Consider an infinitesimal fluid element, its mass would be, 
\begin{equation}
  m = \rho\delta V 
\end{equation}
and the acceleration is, 
\begin{equation}
  \mathbf{a} = \frac{\text{D}\mathbf{v}}{\text{D}t}.
\end{equation}
The two forces acting on the fluid element are the pressure force and the gravitational force. The pressure force is, 
\begin{equation}
  \mathbf{F}_P = - \int_S P\, d\mathbf{S} = -\int_V \nabla P\, dV, 
\end{equation}
which then for an infinitesimal volume becomes,
\begin{equation}
  \mathbf{F}_P = -\nabla P\delta V. 
\end{equation}
The gravitational force for an infinitesimal volume is, 
\begin{equation}
  m\mathbf{g} = \rho\delta V\eqnmarkbox[red]{grav}{\mathbf{g}}. 
\end{equation}
\annotate[yshift=-0.5em]{below}{grav}{$\mathbg{g} = -\nabla\Phi$} 
\hspace{-5.5mm} Bringing everything together,
\begin{definition}[Euler equation]
\begin{equation}
  \rho\frac{\text{D}\mathbf{u}}{\text{D}t} = -\nabla P - \rho\nabla\Phi \quad\quad \frac{\partial\mathbf{u}}{\partial t} + (\mathbf{u}\cdot\nabla)\mathbf{u} = -\frac{1}{\rho}\nabla P - \nabla\Phi. 
\end{equation}
\vspace{-5mm}
\end{definition} 

Currently we have 6 unknowns, $\rho, \mathbf{u}, P, \Phi$, and 3 equations, the continuity equation, the Euler equation, and the Poisson equation,
\begin{definition}[Poisson equation]
\begin{equation}
  \nabla^2\Phi = 4\pi G(\rho + \rho_\text{ext}). 
\end{equation}
\vspace{-5mm}
\end{definition}

\section{Equation of state}
The equation of state is a relation between the pressure, density, and temperature of a fluid. For an ideal gas, 
\begin{equation}
  P = \frac{\rho k T}{\eqnmarkbox[red]{atmweight}{\mu} \eqnmarkbox[blue]{proton}{m_p}}, 
\end{equation}
\annotate[yshift=-0.5em]{below, left}{atmweight}{Atomic weight}
\annotate[yshift=-0.5em]{below, right}{proton}{Proton mass}\\


A gas can be assumed to be \textsf{baroclinic}, $P = P(\rho, T)$, or \textsf{barotropic}, $P = P(\rho)$. Typical situations are \textsf{isotropic fluids}, $P\propto\rho$, and \textsf{adiabatic fluids}, $P\propto\rho^\gamma$. 
\section{Energy equation} 
Firstly, the total energy density of a system is, 
\begin{equation}
  \varepsilon = \rho\left(\eqnmarkbox[red]{kinetic}{\frac{1}{2}\mathbf{u}^2} + \eqnmarkbox[blue]{specintern}{\mathcal{U}}+\eqnmarkbox[PineGreen]{grav}{\Phi}\right),
\end{equation}
\annotate[yshift=-0.5em]{below,left}{kinetic}{Kinetic energy density}
\annotate[yshift=-0.5em]{below}{specintern}{Specific internal energy, $\mathcal{U} = U/M$}
\annotate[yshift=1em]{above}{grav}{Gravitational potential energy density}\\

Taking the derivative, 
\begin{equation}
  \frac{\text{D}\varepsilon}{\text{D}t} = \frac{\varepsilon}{\rho}\frac{\text{D}\rho}{\text{D} t} + \rho\frac{\text{D}}{\text{D}t}\left(\frac{1}{2}\mathbf{u}^2 + \mathcal{U} + \Phi\right). 
\end{equation}
The derivative of the spefic kinetic energy is,
\begin{equation}
\frac{\text{D}}{\text{D}t}\left(\frac{1}{2}\mathbf{u}^2\right) = \mathbf{u}\cdot\frac{\text{D}\mathbf{u}}{\text{D}t} = -\frac{\mathbf{u}}{\rho}\cdot\nabla P - \mathbf{u}\cdot\nabla\Phi.
\end{equation} 
Using the first law of thermodynamics,
\begin{equation}
  d\mathcal{U} + P\,d\mathcal{V} = \delta Q,
\end{equation}
and taking the derivative, 
\begin{equation}
  \frac{\text{D}\mathcal{U}}{\text{D}t} + P\frac{\text{D}\mathcal{V}}{\text{D}t} = \dot Q.
\end{equation}
The specific volume is $\mathcal{V} = 1/\rho$, so the derivative is, 
\begin{equation}
  \frac{\text{D}\mathcal{V}}{\text{D}t} = -\frac{1}{\rho^2}\frac{\text{D}\rho}{\text{D}t} = \frac{1}{\rho}\nabla\cdot\mathbf{u}. 
\end{equation}
The derivative of the gravitational potential energy is,
\begin{equation}
  \frac{\text{D}\Phi}{\text{D}t} = \mathbf{u}\cdot\nabla\Phi + \frac{\partial\Phi}{\partial t}. 
\end{equation} 
Putting everything together,
\begin{definition}[Conservation of energy density]
\begin{equation}
  \begin{aligned}
    \frac{\text{D}\varepsilon}{\text{D}t} &= -\varepsilon\nabla\cdot\mathbf{u} - \mathbf{u}\cdot\nabla P - P\nabla\cdot\mathbf{u} + \rho\dot Q + \rho\eqnmarkbox[red]{excl}{\frac{\partial\Phi}{\partial t}},\\
    \frac{\partial \varepsilon}{\partial t} + \nabla\cdot((\varepsilon + P)\mathbf{u}) &= \rho\dot Q.
  \end{aligned}
\end{equation}
\annotate[yshift=-0.5em]{below, left}{excl}{Excluded as being negligible}
\vspace{-5mm}
\end{definition}
The physical meaning of this equation can be unveiled by integrating the equation, using the divergence theorem,
\begin{equation}
  \frac{\partial}{\partial t}\int_V \varepsilon\,dV + \eqnmarkbox[blue]{flux}{\oint_S \varepsilon\mathbf{u}\cdot d\mathbf{S}} + \eqnmarkbox[red]{workdone}{\oint_S P\mathbf{u}\cdot d\mathbf{S}} = \int_V \rho\dot Q\,dV. 
\end{equation}
\annotate[yshift=-0.5em]{below}{workdone}{Work done by the fluid $dW/dt$}
\annotate[yshift=-0.5em]{below, left}{flux}{Energy flux}\\

\section{Heat exchange}
\subsection{Conduction}
Thermal conduction is the transfer of heat from a region of high temperature to a region of low temperature. Mathematically it can be assumed that the flux of thermal conduction is linearly dependent on the temperature gradient, 
\begin{equation}
  \begin{aligned}
    \mathbf{q} &= -\kappa\nabla T,\\
               &= -\kappa_\text{Sp} T^{5/2}\nabla T. 
  \end{aligned}
\end{equation}
If there is a magnetic field present, the \textsf{supression factor $f$} has to be taken into account, and is inserted in the beginning of the formula. It is usually between $0.01-0,2$, making it a non-negligible factor.\\
Putting everything into the energy equation,
\begin{equation}
  \frac{\partial \varepsilon}{\partial t} + \nabla\cdot((\varepsilon + P)\mathbf{u}) = \rho \dot Q - \nabla\cdot\mathbf{q}. 
\end{equation}

\subsection{Convection}
