\chapternotnumbered{Introduction} \label{ch:Introduction}
\section*{Interstellar Medium}
Components of the interstellar medium (ISM) are 
\begin{itemize}
  \item \textsf{Cold neutral medium (CNM)} -- $T \sim 70$ K
  \item \textsf{Warm neutral medium (WNM)} -- $T \sim 8000$ K
  \item \textsf{HII regions} -- photoionized gas
  \item \textsf{Warm ionized medium (WIM)} -- photoionized gas
  \item \textsf{Hot intercloud medium (HIM)} -- collisionally ionized gas
  \item \textsf{Molecular gas} -- mostly in molecular clouds
\end{itemize} 

\begin{table}[h]
    \centering
    \begin{tabular}{lcccccc}
        \toprule
        Property & CNM & WNM & HII reg & WIM & HIM & Mol \\
        \midrule
        Density (cm$^{-3}$) & 30 & 0.3 & $10 - 10^4$ & 0.1 & $10^{-3}$ & $> 10^2$ \\
        Temperature (K) & $30 - 120$ & 8000 & \multicolumn{2}{c}{$\approx 10^4$} & $> 10^6$ & $\sim 10$ \\
          Thermal speed (km s$^{-1}$) & $\sim 1$ & 8 & \multicolumn{2}{c}{$\sim 10$} & $> 100$ & $\sim 0.1$ \\
          Observed speed (km s$^{-1}$) & few$-$10 & $8 - 15$ & \multicolumn{2}{c}{$15 - 25$} & - & $\sim 1$ \\
        Total Mass ($10^9 M_\odot$) & 2 & 2 & 0.1 & 1 & $< 0.1$ & 1 \\
        Scale-height (kpc) & 0.1 & $0.2 - 0.3$ & - & 1 & 3 & 0.06 \\
        Maximum Radius (kpc) & $\lesssim 25$ & 25 & $\sim 15$ & - & - & $\sim 10$ \\
        Filling factor & 0.01 & 0.25 & 0.02 & 0.25 & $\approx 0.5$ & 0.001 \\
        Main observables & \multicolumn{2}{c}{21-cm} & \multicolumn{2}{c}{recomb/forbid} & X-rays & CO \\
        \bottomrule
    \end{tabular}
    \caption{Properties of different interstellar medium phases.}
    \label{tab:ism_properties}
\end{table}

Cold neutral gas is made of clouds, 90\% H, 10\% He, and trace elements. The photoionized gas is mostly HII regions, and can be seen through emission of \textsf{recombination} ($\text{Ly}\alpha, \text{H}\alpha, \text{H}\beta$, ...), \textsf{forbidden lines} ([OII], [OIII], [NII], [SII], ...), and \textsf{bremsstrahlung}. Hot gas is collisionally ionized and emits X-rays. Hot gas extends to hunderds of kiloparsecs making it difficult to observe. 
\\

The thermal velocity is given by 
\begin{equation}
  v_{\text{th}} = \sqrt{\frac{kT}{m}} 
\end{equation}\\

A magnetic field can be detected by 
\begin{itemize}
  \item \textsf{Zeeman effect} -- splitting of spectral lines
  \item \textsf{Synchrotron radiation} -- non-thermal radiation
  \item \textsf{Faraday rotation} -- rotation of polarization angle
  \item \textsf{Polarization of stellar light} 
  \item \textsf{Dust polarization} 
\end{itemize}

The magnetic field is important for gas dynamics, cloud stability, and shocks.

\section*{Equilibriums in Hydrodynamics}
The medium can be divided into three phases:
\begin{itemize}
  \item \textsf{Cold phase} -- $T \lesssim 100$ K 
  \item \textsf{Warm phase} -- $T \sim 10^4$ K 
  \item \textsf{Hot phase} -- $T \gtrsim 10^6$ K
\end{itemize}
and the pressures are roughly equivalent in the three phases. \textsf{Thermal equilibrium} is only a \hyellow{reasonable approximation locally}, not globally for the ISM. This is contrasted by the \textsf{pressure equilibrium} which is often \hyellow{fulfilled globally} in the ISM.\\
\textsf{Hydrostatic equilibrium} is the balance between the pressure and gravitational force. It is important for molecular clouds, galaxy discs, hot halos, stars, ...\\

\section*{Collisional Processes}
In ideal fluids there are two types of collisions, \textsf{elastic} (change in $\mathbf{v}_p$, and energy is conserved), and \textsf{inelastic collisions} (with the transfer of energy). The mean free path is the average distance between collisions, 
\begin{Note}[Mean free path]
\begin{equation}
  \ell = \frac{1}{\eqnmarkbox[blue]{n}{n} \eqnmarkbox[red]{sigma}{\sigma}},
\end{equation}
\annotate[yshift=-0.5em]{below, left}{n}{Number density}
\annotate[yshift=-0.5em]{below}{sigma}{Cross-section}
\vspace{5mm}
\end{Note}
and the \textsf{average time between collisions} is 
\begin{equation}
  \tau = \frac{\ell}{v_{\text{th}}} = \frac{1}{n\sigma v_{\text{th}}} \sim 10^4 \left(\frac{n}{\si{cm^{-3}}}\right)^{-1}\si{s} \text{ \hspace{0.3cm}  (\textsf{WIM})}. 
\end{equation} 
The shortest inelastic collision in WIM is the ionization of $\text{O}^+$ to OII, and the excitation time is, 
\begin{equation}
  \tau_{\text{exc}}(\text{O}^+) \sim 10^{11} \left(\frac{n_{\text{O}^+}/ n_e}{10^{-3}}\right)^{-1}\si{s}.
\end{equation}
\newpage
In CNM, the average collision time is $\sim 10^8\, \text{s}$, mostly due to the C$^+$. The velocities as distributed as a \textsf{Maxwellian distribution},
\begin{definition}[Maxwellian distribution]
\begin{equation}
  f(v)\,dv = 4\pi \left(\frac{m}{2\pi kT}\right)^{3/2} v^2 \exp\left(-\frac{mv^2}{2kT}\right)\,dv. 
\end{equation}
\vspace{-0.5cm}
\end{definition}
Using the distribution, the following characteristing speeds can be derived,
\begin{Note}[Peak velocity]
\begin{equation}
  v_{\text{peak}} = \sqrt{\frac{2kT}{m}},
\end{equation}
\end{Note}
\begin{Note}[RMS velocity]
\begin{equation}
  v_{\text{rms}} = \sqrt{\frac{3kT}{m}}. 
\end{equation}
\end{Note}
\begin{Note}[Mean velocity]
\begin{equation}
  \langle v \rangle = \int_0^\infty v f(v)\,dv =  \sqrt{\frac{8kT}{\pi m}},
\end{equation}
\end{Note}

